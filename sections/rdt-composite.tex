% file: section/rdt-composite.tex

\section{Replicated Data Types with Composite Operations}

\subsection{Specification}	\label{ss:rdt-composite-spec}

We consider a composite operation of a replicated data type $\tau$ 
in the form of $C = A \oplus B$, 
where $A$, $B$, and $C$ are different objects of type $\tau$.
\marginnote{\question{Generalize to different data types?}}

Following~\cite{Burckhardt:POPL14}, we specify the semantics of 
a composite operation $A \oplus B$ of a replicated data type $\tau$
by a function $\F_{\tau}$ that determines the return value of $\oplus$ 
based on prior operations performed on the two objects involved (i.e., $A$ and $B$).
However, $\F_{\tau}$ for a composite operation $\oplus$ 
takes as parameters two, not one as in~\cite{Burckhardt:POPL14}, 
\emph{operation contexts}, one on each object involved.
\marginnote{\note{Partial operation context~\cite{Gotsman:ESOP15}}}

\begin{definition}[Product of Operation Contexts]
  Consider two operation contexts for the same replicated data type $\tau$:
  \begin{align}
    \C_{A} = (E_A, \op_{A}, \vis_{A}, \ar_{A}) \\
    \C_{B} = (E_B, \op_{B}, \vis_{B}, \ar_{B})
  \end{align}

  The product $\C = \C_{A} \times \C_{B}$ of $\C_{A}$ and $\C_{B}$ 
  is also an operation context defined as $\C = (E, \op, \vis, \ar)$, where
  \begin{itemize}
    \item $E = E_A \times E_B$
    \item $\op = \op_{A} \sqcup \op_{B}$
    \item $\vis = \vis_{A} \times \vis_{B}$
    \item $\ar = \ar_{A} \times \ar_{B}$
  \end{itemize}
\end{definition}

\begin{definition}
  \begin{align}
    \F_{\tau}(\oplus, \C_{A}, \C_{B}) 
    = \F_{\tau}(\oplus, \C_{A} \times \C_{B})
  \end{align}
\end{definition}
